\documentclass[11pt,a4paper,titlepage]{article}
% Could add twoside if wanted. 
%%%%%%% Standard settings %%%%%%%
\usepackage{amsmath}
\usepackage{amssymb}
% \usepackage[nottoc, numbib, notlot, notlof]{tocbibind}
\usepackage{fancyhdr}
\usepackage[utf8]{inputenc} 
%\usepackage{mathtools}
\usepackage[danish]{babel}
%\usepackage{microtype}
%\usepackage[svgnames]{xcolor}
%\usepackage{newclude}
\usepackage{siunitx}
%\usepackage{booktabs}
\usepackage{wrapfig} 
\usepackage[subfigure]{tocloft}
\usepackage{todonotes}
%\usepackage[disable]{todonotes}
\usepackage{lastpage} 
%\usepackage{lipsum,lastpage}  
\usepackage{hanging}
\usepackage{tabularx}
\setlength{\headheight}{15pt}
\usepackage{filecontents}
\usepackage{longtable}
\usepackage{footnote}
\setcounter{tocdepth}{2} % Antal lag i indholdsfortegenlse.
\setcounter{secnumdepth}{5} % Antal lag i sektioner.
\usepackage[toc,page]{appendix} % appendices
\usepackage{gensymb}
\usepackage{chngcntr}
\counterwithin{figure}{section}
\counterwithin{table}{section}
\counterwithin{equation}{section}

% % Tosidet margin:
% \usepackage[margin=2.5cm, inner =3cm,outer =2cm]{geometry}
% % Enkeltsidet magin: 
\usepackage[margin=2.5cm]{geometry}


\usepackage{paralist}
  \let\itemize\compactitem
  \let\enditemize\endcompactitem
  \let\enumerate\compactenum
  \let\endenumerate\endcompactenum
  \let\description\compactdesc
  \let\enddescription\endcompactdesc
  \pltopsep=1pt
  \plitemsep=1pt
  \plparsep=1pt

%%%%%%% FIGURES %%%%%%%

\usepackage{graphicx} % bruges til MATLAB figurer samt formatet .eps
\usepackage{epstopdf} % bruges til automatisk konvertering af .eps-filer til pdf; Ellers kan man ikke kompilere med pdfLaTeX.


%%%%%%% captions settings %%%%%%%
\usepackage[round]{natbib}
\bibliographystyle{plainnat}

%%%%%%% captions settings %%%%%%%
\usepackage[margin=1cm]{caption}
\captionsetup{
  font=footnotesize,
  width=0.75\textwidth
}
\captionsetup[subfigure]{
  font=footnotesize,
  width=0.8\textwidth
}
\usepackage{subcaption} %subfigures


%%%%%%% implementering af code %%%%%%%
\usepackage{listings}
\lstset{frame=tb,
tabsize=4,
language=C++,
captionpos=b,
tabsize=3,
frame=false,
numbers=left,
numberstyle=\tiny,
numbersep=5pt,
breaklines=true,
showstringspaces=false,
basicstyle=\ttfamily,
% identifierstyle=\color{magenta},
keywordstyle=\color[rgb]{0,0,1},
commentstyle=\color{ForestGreen}\ttfamily,
morecomment=[l][\color{brown}]{\#},
stringstyle=\color{red}
}
%Figurtekst under kode udsnit
\renewcommand\lstlistingname{Code snippet}

%%%%%%% Diverse %%%%%%%
\usepackage{icomma}
\usepackage{pdfpages}
\usepackage[pdftex,
 hyperfigures=true,
 pdfauthor={},
 pdftitle={},
 pdfsubject={},
 pdfkeywords={},
 hidelinks,
 plainpages=false,
 pdfpagelabels,
 unicode]{hyperref}
 
\usepackage{textcomp,gensymb} % Tilføjer et grader-tegn i mat-mode

\newenvironment{changemargin}[2]{%
\begin{list}{}{%
\setlength{\topsep}{0pt}%
\setlength{\leftmargin}{#1}%
\setlength{\rightmargin}{#2}%
\setlength{\listparindent}{\parindent}%
\setlength{\itemindent}{\parindent}%
\setlength{\parsep}{\parskip}%
}%
\item[]}{\end{list}}