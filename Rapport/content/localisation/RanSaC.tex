\subsubsection{Feature Detection}
To detect the features of the robots whereabouts it was decided to implement the RanSaC algorithm.
During tests of this implementation, it was found that RanSaC
had tendencies to detect two similar lines next to each other for the same features,
when this feature was represented by many points in the sample.
Therefore, it was decided to implement the RanSaC with a merge function.
The angle and distance to the two lines representing the same feature were then averaged to give one line,
represented by a polar vector to the point on the line that is closest to the origin.

%Furthermore a variation of RanSaC was made. This implementation was then compared to the traditional RanSaC method, using the same parameters, on a dataset collected from the 2D laser scanner while performing the UMB mark test (driving in a $ 1 x 1 m$ square). This gave 1761 unique data set samples from the 2D scanner in the dataset used to test.
%The  the lines are plotted in matlab against the data points from the 2D scanner. The number of lines plotted where no features should be found was then counted for the whole data set and compared.


It was decided to represent all lines as polar lines, $ (R, \theta) $,
where $R$ is the shortest distance to the line from the origin and $\theta$ is the angle to the x-axis.
For simplicity, all lines are standardized to conform with $ 0 \leq \theta < 2\pi $ and $ R \geq 0 $.
The modified version of RanSaC was implemented as follows:

\begin{enumerate}
\item Copy the data from the 2D laser scanner into a list, $ \Lambda_{points} $ ,
of polar points $ (R,\theta) $, excluding all points with $ R = 0 $ which correspond to points not visible to the scanner.
\item Copy two unique random points into $ \Lambda_{sample} $ and find the line, $ \lambda_{init} $, passing through them.\label{item:genRandomPoints_LamdaInit}
\item Copy all points within $ \Delta_{max deviation} $ distance to $ \lambda_{init} $ into $ \Lambda_{sample} $.
If the size of $ \Lambda_{sample} $ passes a minimum threshold,
go to item \ref{item:tresholdPassed_RemovePoints}
else go to \ref{item:genRandomPoints_LamdaInit}.
\item Delete all points from $ \Lambda_{points} $) within $ \Delta_{max deviation} $ from $ \lambda_{init} $.
\label{item:tresholdPassed_RemovePoints}
\item Find the best fit line, $ \lambda_{final} $,
passing through the dataset $ \Lambda_{sample} $. Place $ \lambda_{final} $ in the final line set.
\item Until a specific number of RanSaC-loops are done without finding a line
or if the dataset is too small to contain a valid line,
go to item \ref{item:genRandomPoints_LamdaInit},
else continue.
\item Standardize the lines found to fit the specified format and merge lines considered to represent the same feature.
%\item Recompute the line for $ \Lambda_{final} $ and store the line. Go then to item \ref{item:genRandomPoints_LamdaInit}.
%\item Move all data points back to $ \Lambda_{points} $ and go back to \ref{item:genRandomPoints_LamdaInit}. \label{item:lineNotImproved_refillLamdaPoints}
\end{enumerate}

The data points form the laser range scanner which were found to have $ R = 0 $
were removed because they were, defined by the scanner, points for which the object is out of reach.
The parameters for the RanSaC were experimentally, through trial-and-error,
found to fit the robots collected data points as good as possible.
