\section{Localisation}
\subsection{Method}
For the localisation part of the project, it was chosen to use the Kalman Filter. The mobile robot used is a "Nexus Robot - 2WD mobile robot kit 10004", fitted with a laserrange scanner of type "Hokuyo URG-04LX-LG". 

The Kalman filter is an optimal technique for fusing different estimates of positions. \footnote{Reference to siegwart (p. 233)} 
For this section 2 different estimates will be calculated:

\begin{itemize}
	\item Position estimate based on odometry (wheel encoders).
	\item Position estimate based on line features extracted from the laserrange scanner. 
\end{itemize}


The performance of the localization technique was measured using UMB-mark. 
The position before and after following a preprogrammed path through a 1 x 1 [m] square 10 times was compared. 
While testing, the robot only used sensor feedback from the encoders. 
Sensor readings from both the encoders and the laserrange scanner was gathered, and saved for further offline processing. 

Using this data, the final position is calculated using both the encoders and the laserrange scanner. This is done using the Kalman filter. 
%If this offline calculation using both estimates is more accurate than the online estimate using only odometry, then localisation is proved to be effective than odometry. 

\subsubsection{Kalman Filter}
\label{sec:Kalman_filter}
Using the general Kalman filter follows these 5 steps: 
\todo[inline, author=Michael]{This stuff is described pretty well in Siegwart. Read chapter 5.2.4 and 5.6.3.2 and you're good to go.}
\begin{enumerate}
	\item Predict robot position based on old location (using model of robot and control input).
	\item Observe sensor measurements. 
	\item Predict sensor measurements. 
	\item Matching predicted and observed sensor measurements. 
	\item Estimate position by applying Kalman filter.
\end{enumerate}

To simplify the task, it was decided to skip step one and instead rely completely on data from the sensors. 
\todo[inline, author=Michael]{This is also necessary to use the tests done so far (28/11 2014), because we have no knowledge about the internal control inputs during our tests.}

